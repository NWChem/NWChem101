\documentclass[letterpaper,12pt]{article}

\title{NWChem 101: a tutorial for experienced quantum chemists
       \footnote{
Copyright (C) 2009, Jeff R. Hammond. Permission is granted to copy, distribute and/or modify this document under the terms of the GNU Free Documentation License, Version 1.3 or any later version published by the Free Software Foundation; with no Invariant Sections, no Front-Cover Texts, and no Back-Cover Texts. See \texttt{<http://www.gnu.org/copyleft/fdl.html>} for the full text of this license.
       }
}
\author{Jeff R. Hammond \\
        Leadership Computing Facility \\
        Argonne National Laboratory \\
        Argonne, IL 60439 \\
        \texttt{jhammond@mcs.anl.gov}}
\date{Last edited: \today}

\begin{document}

\maketitle

% \begin{abstract}
% 
% \end{abstract}

\section{Introduction}\label{sec:Introduction}

The document (101) introduces the molecular quantum chemistry capability of NWChem in pedagogical manner.  It should not replace the \textit{NWChem User Manual} (UM), but rather supplement it.  The UM is organized by functionality and documents all options while 101 is organized from the most basic features to the most advanced.

\newpage

\subsection{Input File Structure}\label{sec:InputFileStructure}

Figure \ref{fig:WaterSimple} is a basis NWChem input file which introduces a number of key elements.  The important input elements are described briefly in Table~\ref{tab:BasicInput}.

The \texttt{echo} should be considered essentially since it is the best way to prevent data rot.  It is common to modify input files throughout a project, thus the only way to preserve the exact settings used for a given calculation is to retain the input file with the corresponding output.  It is tedious to extract the original input settings from the output file, and in some circumstances, these may not be available at all (e.g. when \texttt{print low} or \texttt{print none} are used).

The \texttt{start} directive is also essentially since it controls the names of files that NWChem will create.  It is critical that the RTDB file be unique, as running two jobs simultaneously with the same prefix will frequently result in an error or incorrect behavior.  Using \texttt{title} is optional and less useful than \texttt{echo} with a commented input file.

Comments can be introduced to input files using the \texttt{\#} symbol.  These comments will be echoed along with the rest of the input file.  One can comment after a regular input line as demonstrated in the example above.  Suggested comments include: (1) a description of where the input geometry came from (e.g. ``this is the B3LYP/6-31G optimized geometry''), (2) the citation output from the EMSL Basis Set Exchange (\texttt{bse.pnl.gov}), when this is the source of the basis set input, and (3) annotation of undocumented options users, system-specific paramters and \texttt{set} directives (described in Section~\ref{sec:SetDirectives}).

\begin{figure}
    \caption{A very simple input file.}
    \label{fig:WaterSimple}
    \begin{verbatim}
 1   echo
 2   start water
 3   title "a simple calculation of water"
 4   # COMMENT: these options are system specific
 5   memory total 2000 mb
 6   permanent_dir /home/jeff/scratch/nwchem
 7   scratch_dir /home/jeff/scratch/nwchem
 8   geometry angstroms # angstroms are the default units
 9     O   0.00000000  0.00000000   0.11726921
10     H   0.75698224  0.00000000  -0.46907685
11     H  -0.75698224  0.00000000  -0.46907685
12   end
13   basis
14   * library 6-31G*
15   end
16   task scf energy
    \end{verbatim}
\end{figure}

\begin{table}[!hp]
    \label{tab:BasicInput}
    \caption{Important input directives to be used in any type of calculation.  Line numbers refer to Figure~\ref{fig:WaterSimple}.  See Section~\ref{sec:InputFileStructure} more detail descriptions.}
    \begin{tabular}{lcl}
        \hline\hline
        directive               & line  & purpose \\
        \hline
        \texttt{echo}           & 1   & prints the input file to output stream \\
        \texttt{start}          & 2   & initializes the RTDB and other files using given prefix \\
        \texttt{title}          & 3   & this title will be printed with each module's output \\
        \texttt{\#}             & 4,8 & stops parsing at the \texttt{\#} symbol \\
        \texttt{memory}         & 5   & configures memory usage \\
        \texttt{permanent\_dir} & 6   & the directory where persistent files will be stored \\
        \texttt{scratch\_dir}   & 7   & the directory where scratch files will be stored \\
        \texttt{geometry}       & 8   & input block for the molecular geometry \\
        \texttt{basis}          & 13  & input block for the basis set \\
        \texttt{task}           & 16  & NWChem will execute the directive given here \\
        \hline\hline
    \end{tabular}
\end{table}

\newpage

\subsection{Geometry Input}\label{sec:GeometryInput}

Geometry input may be specified in both cartesian and zmatrix format.  Point-group symmetry will be detected automatically but in some cases it must be set manually.  First, not all modules implement non-degenerate point-group symmetry, in which case the symmetry must be set to $D_{2h}$ or one its subgroups (see Figure~\ref{fig:NeonGeometry}).  Some modules detect non-degenerate point-groups and disable symmetry altogether so it recommended that one set the appropriate symmetry manually for TCE jobs.  In other cases, it is desirable to use symmetry to generate a molecule from a fragment.  An example is given in Figure~\ref{fig:C60geometry}.  Finally, an example of a non-trivial zmatrix geometry input is given in Figure~\ref{fig:HexafluorobenzeneZMatrix}.

The UM is an excellent resource for the geometry input block so we will not elaborate further on this topic.

\begin{figure}
    \caption{The geometry input for an atom with a forced reduction in symmetry.}
    \label{fig:NeonGeometry}
    \begin{verbatim}
 1   geometry
 2     symmetry d2h
 3     Ne 0.0 0.0 0.0
 4   end
    \end{verbatim}
\end{figure}

\begin{figure}
    \caption{Geometry input for C$_{60}$ using symmetry completion.}
    \label{fig:C60geometry}
    \begin{verbatim}
 1   geometry print xyz units bohr
 2    symmetry d2h
 3   C  0.0000000   6.5776597   1.3147448
 4   C  2.2251790   5.7277170   2.6899811
 5   C  1.3752363   4.3524807   4.9151601
 6   C  4.3524807   4.9151601   1.3752363
 7   C  2.6899811   2.2251790   5.7277170
 8   C  4.9151601   1.3752363   4.3524807
 9   C  5.7277170   2.6899811   2.2251790
10   C  6.5776597   1.3147448   0.0000000
11   C  1.3147448   0.0000000   6.5776597
12   end
    \end{verbatim}
\end{figure}

\begin{figure}
    \caption{Geometry intput for hexafluorobenzene using the zmatrix format.}
    \label{fig:HexafluorobenzeneZMatrix}
    \begin{verbatim}
 1   geometry units bohr
 2      zmatrix
 3      X
 4      C 1 RXC
 5      C 1 RXC 2 A60
 6      C 1 RXC 3 A60 2 D180
 7      C 1 RXC 4 A60 3 D180
 8      C 1 RXC 5 A60 4 D180
 9      C 1 RXC 6 A60 5 D180
10      F 1 RXF 2 A60 7 D180
11      F 1 RXF 3 A60 2 D180
12      F 1 RXF 4 A60 3 D180
13      F 1 RXF 5 A60 4 D180
14      F 1 RXF 6 A60 5 D180
15      F 1 RXF 7 A60 6 D180
16      constants
17        RXC    2.63428
18        RXF    5.14195
19        A60   60.00
20        D180 180.00
21      end
22   end
    \end{verbatim}
\end{figure}

\newpage

\subsection{Basis Set Input}\label{sec:BasisSetInput}

The two best ways to specify the basis set in an input are to (1) use the NWChem basis set library (NWL) and (2) copy the input generated by the EMSL Basis Set Exchange (BSE).  While the NWChem basis set library is extensive, very recent or highly specialized basis sets must be obtained from the BSE.  Common basis sets in the NWL include those from the Pople series (e.g. 6-31G*), the Dunning series (e.g. cc-pVDZ to d-aug-cc-pV6Z for many elements) and specialized basis sets such as Roos ANO DZ, Sadlej pVTZ (POL), Ahlrichs, DGauss and DeMon fitting basis sets, and many basis sets for heavy elements (e.g. LANL2DZ and CRENBL ECP).  For

\begin{figure}
    \caption{Specification of the three types of basis sets used in NWChem.  One must specify the atomic orbital basis set (\texttt{"ao basis"}) for all calculations, while \texttt{"ri-mp2 basis"} and \texttt{"cd basis"} are used for RI-MP2 and DFT with density-fitting, respectively.}
    \label{fig:BasisLibrary}
    \begin{verbatim}
 1   basis "ao basis" cartesian
 2     C library 6-31G*
 3   end
 4   basis "ri-mp2 basis" spherical print
 5     * library "cc-pVTZ-RI"
 6     * library "aug-cc-pVTZ-RI_diffuse"
 7   end
 8   basis "cd basis"  spherical
 9     * library ahlrichs_coulomb_fitting
10   end
    \end{verbatim}
\end{figure}

\newpage

\subsection{Task Input}\label{sec:TaskInput}

\newpage

\section{Basis Input Files}\label{sec:BasisInputFiles}

\newpage

\subsection{Hartree-Fock (SCF)}\label{sec:HartreeFock}

\newpage

\subsection{Density-Function-Theory (DFT)}\label{sec:DensityFunctionTheory}

\newpage

\subsection{Perturbation Theory (MP2)}\label{sec:PerturbationTheory}

\newpage

\subsection{Coupled-Cluster Theory (CCSD)}\label{sec:CoupledClusterTheory}

\newpage

\subsection{Multi-Configuration SCF (MCSCF)}\label{sec:MultiConfigurationSCF}

\newpage

\section{Advanced Input Files}\label{sec:AdvancedInputFiles}

\newpage

\subsection{Multiple Tasks}\label{sec:MultipleTasks}

\newpage

\subsection{Converging SCF/DFT}\label{sec:ConvergingSCF}

\newpage

\subsection{Effective Core Potential (ECP) Basis Sets}\label{sec:ECPBasisSets}

\newpage

\subsection{Custom Density Functionals}\label{sec:CustomDensityFunctionals}

\newpage

\subsection{Undocumented Options}\label{sec:UndocumentedOptions}

\newpage

\subsection{Set Directives}\label{sec:SetDirectives}

\newpage

\section{TCE Input Files}\label{sec:TCEInputFiles}

\newpage

\end{document}
